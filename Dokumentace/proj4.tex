\documentclass[a4paper, 11pt]{article}
\usepackage[left=2cm,text={17cm, 24cm},top=3cm,left= 2cm]{geometry}
\usepackage[IL2]{fontenc}
\usepackage[czech]{babel}
\usepackage[utf8]{inputenc}
\usepackage{url}
\providecommand{\uv}[1]{\quotedblbase #1\textquotedblleft}
\DeclareUrlCommand\url{\def\UrlLeft{<}\def\UrlRight{>} \urlstyle{tt}}


\begin{document}
\pagestyle{empty}
\begin{center}
\Huge
\textsc{Vysoké učení technické v Brně}\\
\huge
\textsc{Fakulta informačních technologií}\\
\LARGE
\vspace{\stretch{0.382}}
Typografie a publikování - 4. projekt\\ \Huge Bibliografická citace
\vspace{\stretch{0.618}}
\end{center}
{
\LARGE \hfill
Vojtěch Meluzín \\
\today \hfill
Matěj Mlejnek
}

\newpage
\setcounter{page}{1}
\pagestyle{plain}

\section{Písmo}
\uv{Pro design tiskoviny, typografickou úpravu, je písmo základ.} \cite{jak_publikovat_napocitaci}
U určité příležitosti se nehodí použít stejné písmo jako u jiné. Proto pro různé typy 
dokumentů používáme specifické styly vhodné pro daný typ. \cite{dipl_martin_cerny, typografia_cz}\\ 
V následujících podsekcích je vypsáno pár atributů písem s příklady, u kterých je napsán příkaz pro~implementaci do \LaTeX u. 
\subsection{Rodina}
Značí styl pro skupinu od sebe odvozených písem. Mezi základní patří: \\ 
\verb|\textrm| \hspace{0.5cm} \textrm{Roman family} - latinka;\\
\verb|\textsf| \hspace{0.5cm} \textsf{Sans serif family} - bezpatkové písmo;\\
\verb|\textttt| \hspace{0.3cm} \texttt{Typewriter family} - strojopisné písmo;
\cite{latex_kompletni_pruvodce, typograficky_manual}


\subsection{Varianty}
Tvary písem lze dále ovlivnit použitím jejich variant. Mezi základní patří: \\
\verb|\itshape| \hspace{0.5cm} \itshape{kurzíva;} \\
\verb|\slsshape| \hspace{0.28cm} \slshape{naklonění;} \\
\verb|\scshape| \hspace{0.5cm} \scshape přepnutí na velká písmena a kapitálky; \normalfont
\cite{latex_kompletni_pruvodce, typograficky_manual}

\subsection{Váha}
Váha písma určuje šířku neboli tloušťku písma. Lze použít např: \\
\verb|\mdseries| \hspace{0.5cm} přepnutí na střední(defaultní) tloušťku; \\
\verb|\bfseries| \hspace{0.5cm} přepnutí do \bfseries tučného \normalfont písma;
\cite{latex_kompletni_pruvodce, typograficky_manual}

\section{\TeX}
\TeX \- je typografický systém původně určený zejména pro kvalitní sazbu knih s velkým množstvím matematiky. \TeX \- byl Roku 1983 světu darován k volnému používaní jeho tvůrcem D. E. Knuthem. Systém díky kvalitnímu zpracování 
nepotřeboval od vydání žádnou zásadní změnu ve své koncepci. \\Díky zvyšující se oblíbenosti existuje dnes ohromné množství nadstaveb, knihoven a různých nástrojů, které nám výrazně ujednodušují práci při sepisování nejen odborných publikací.
\cite{dipl_martin_cerny}

\section{Bib\TeX}
Bib\TeX \- je nástroj pro \LaTeX \- pomocí něhož se dájí vytvářet odkazy na použitou literaturu.    \\
Seznam citací se vytváří pomocí příkazu \verb|\bibliography{jméno souboru}|. Tento soubor obsahuje jednotlivé položky a informace, které jsou pro daný typ zdroje nutné a volitelné. 
Jednotlivé citace do~textu vkládáme pomocí příkazu \verb|\cite{název navěští}|.
\cite{dipl_michal_janda, bibtex_alex}

\section{DPI - Dots Per Inch}
DPI je hodnota jejíž velikost určuje počet bodů, vyobrazených na jednom palci ($1~palec = 2,54~cm$) z anglického slova \uv{inch}. Velikost nás zajímá zejména při tisku nebo skenování.
Pro tisk obyčejného textu nám postačí 100 DPI. U obyčejných spíše méně kvalitnějších obrázků se považuje za minimum 150-200 DPI. U Fotek a podrobnějších objektů se za vhodný standard považuje 300 DPI.
\cite{computer_casopis, vsellis, repronis_foto}

\pagebreak
\newpage

\bibliographystyle{czechiso}
\def\refname{Literatura}
\bibliography{proj4}

\end{document}